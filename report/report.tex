\documentclass[12pt, letterpaper, twoside]{article}
%\usepackage{epstopdf}
%\usepackage[pdftex]{graphicx}
\usepackage[left=.8in,top=.8in,right=.8in,bottom=.8in,nohead,nofoot,columnsep=20pt]{geometry}
\usepackage{setspace}
\usepackage[small,compact]{titlesec}
\usepackage{subfigure}
\usepackage{multicol}
%\usepackage{multirow}
\usepackage{textcomp}
\usepackage{mathtools}
\usepackage{amsmath}
\usepackage[hyphens]{url}
\usepackage{float}
\usepackage{fancyhdr}
\setlength{\headheight}{14.5pt}
\setlength{\footskip}{11pt}
\pagestyle{fancy}
\lhead{}
\chead{}
\rhead{}
\rfoot{}

\title{Title goes in report.tex}
\author{Bonnie Eisenman, Michael Kranch and Anna Kornfeld Simpson}
\date{COS 597E Software Defined Networks \\ 14 January 2014}

\begin{document}

\maketitle

\begin{spacing}{1.0}

\begin{abstract}
Here's some space for our abstract, which also is in report.tex ... compile this with make in the report directory.  The makefile runs the latex compiler pdflatex 3 times in order to get the references and citations correct; if it says you have undefined references in the output at the end scroll up in the compiler output to find them!
\end{abstract}

\begin{multicols}{2}

\section{Introduction}
We probably want to introduce things, right now that's in report.tex.  We might want to cite something in the introduction, we can do that using \begin{verbatim}{\cite{refname}\end{verbatim} where the \emph{refname} is defined in bibliography.tex in the \begin{verbatim}{\bibitem}\end{verbatim} command.  For example citing dpkt looks like this: \cite{dpkt}.

\section{Background}

\subsection{Software Defined Networking}

\subsection{Security in SDN}

\section{Creating a Software Defined Switch}

\subsection{OpenFlow Protocol}

\subsection{Initiating a Connection to a Controller}

\subsection{Push verses Pull Connections}

\label{fake}
I'm making up sections and order... there's a file called fake.tex where we can write about faking a switch. The verbatim environment is useful for preformatted text - there's an example in the introduction!
In this section, we present our roque switch and 

\subsection {OpenFlow Protocol}
OpenFlow is the protocol used to communicate between the controller and its switches. An OpenFlow packet header is simply an 8 byte packet with the first byte used to communicate version, the second byte for type, third and fourth byte for message length followed by a four byte Transaction ID (see Figure 1). The type is of a subset of 19 possible types, most of which are of type request (e.g. 16 = Stats Request) with the subsequent reply (e.g. 17 = Stats Reply). There are couple key point to note when dealing with the OpenFlow protocol. First, the transaction ID is used to correlate incoming OpenFlow messages with their appropriate responses much in the same way TCP uses SYN and ACK flags. For example, a features reply will (generally) not be accepted by the controller unless it contains the corresponding transaction ID from the features request. This protocol is also a two-way communication scheme and not simply switch replies to controller request. Several message types, to include packet input events, are switch initiated communications to the controller.  

\begin{figure}
  \includegraphics[width=\linewidth]{openflowProtocol.png}
  \caption{OpenFlow Protocol Header Format \cite{protocol}}
  \label{fig:protocol}
\end{figure}

\subsection {Controller Connection Sequence}
While the specific initiation sequence varies by controller, there are several required command to initiate a switch to controller connection\footnote{We specifically tested our roque switch on the Pox, Ryu and OpenDaylight controllers. Based on our testing and the OpenFlow specifications, we believe all controllers require this small subset of initialization commands but differences might exist based on implementation}. The switch to controller connection simply starts by opening a TCP connection to the controller on the configured port (default 6634)\footnote{As previously discussed, there is no authentication for this connection supported by any tested controller outside of certificate authentication only utilized with OpenVSwitch}.  After opening the connection, the switch sends a Openflow Hello message (0) to the controller. The controller then responds with a Hello message and a Features Request (5) message to which the switch responds with a  Features Reply (6). The controller then sends a Set Config message (9) that does not trigger a reply. Ryu follows this message with a Barrier Request (18), and OpenDaylight follows it with a  Get Config Request (7) to ensure switch configs are set appropriately. OpenDaylight also sends a Flow Modification message (14) to delete any previously installed flows on the switch.

The above commands are all that is needed for the initial controller to switch connection. The switch then proceeds to send several link layer neighbor discovery protocol messages as Packet Input Notification (10) messages to the controller including  Neighbor Solicitation and Router Solicitation messages. The switch also sends s Multicast Listener Report (part of the Multicast Listened Discovery Protocol Version 2) to the controller. Finally, the switch begins periodically (approximately every 3 seconds) sending an OpenFlow Echo Request (2) to the controller as a form of a keep alive with the controller.

\subsection{Our Rogue Switch Utility}
Our Roque Switch Utility mimics the initial connection sequence to a controller in order to facilitate testing of switch and controller vulnerabilities. Our roque switch utility does not handle the actual routing of traffic and is instead simply used to disrupt controller and switch traffic flow\footnote{A more advanced utility including routing is discussed in our additional attack section as well as our future work}. The bulk of our utility is a OpenFlow message parser that handles and appropriately responds to controller messages in the correct message format. These messages are generally hardcoded mimicked responses from previously captured live switch communication with minimal dynamic pieces (the transaction id is dynamically assigned for instance)\footnote{Both Pox and Ryu do not actual verify that the transaction ID is correct on most messages. We were able to completely connection to both a Pox and Ryu controller simply my resending previously captured packets. Pox only verifies the transaction id on the Barrier Message of the initial config messages, and this reply can be ignored without the controller ending the TCP session}. 


\section{Disrupting an SDN with a Fake Switch}
\label{attacks}
There's a file called attacks.tex for describing our various attacks; we could combine the results of the attacks we implemented here, or put them separately.

\subsection{Executed Attacks} 

We conducted several attacks on various controller utilizing our rogue switch utility. 

 \subsubsection{DOS on the Controller}
We utilized the basic layer 2 mac address learning example from each controller with a modified the hard flow timeout of 1 as our setup\footnote{We used l2\_learning as our Pox controller and l2\_switch as our Ryu controller.}. We also utilize the basic mininet setup (1 switch with 2 hosts) to test the delay caused by our attack on each controller. We would have h1 ping h2 To test the delay cause on the controller, we would simply A more complicated controller (e.g. one that does more than simply add the mac address as a flow) would see its performance decrease more significantly than in our example test cases.

 The first attack we attempted was to see if a single switch could overload the controller by rapidly pushing information. We utilized the basic Echo Request (the switch keep-a-live message) as the OpenFlow message to send to the controller to establish a baseline for testing. Bigger packets, particularly ones that require some sort of computation by the controller application, would increase the performance effect on the controller. We started by sending a Echo Request as quickly as possible to the controller without caring about the response. This attack had little effect of the controller and simply resulted in periodic "TCP Previous Segment Lost" messages as a response from the controller. The issue with this method was we were never reading in the Echo Replies sent from the controller and the controller was simply ignoring our messages because its send buffer was full [I BELIEVE THIS IS CORRECT NOT 100]. We then modified this attack to send a single Echo Request, wait for an Echo Reply and then instantly send another. This attack caused the controller to continually send Echo Replies but at a controller specified rate - we had to wait for a reply before sending another request. As a result, this attack also had little effect on the controller

   To increase the effect of our attack the controller, we created multiple rogue switches and utilized the same attack on the controller. Our testing machine was actually the limiting factor in the number of rogue switches we could spawn, but we utilized 400 switches as the max for comparison purposes. These results (shown in Figure 2) show that we can severely effect a controller's response time simply by utilizing Echo Request. We further optimized our attack by splitting our TCP socket into a distinct listener and sender. We threaded our application so the two actions were not dependent on each other. With the optimized version, a single switch could fairly significantly lower the performance of the switch. We then tested our optimized attack against our test controller and noticed an even more significant result. When testing with 10 switches, the roque switches 
 too
By sending requests to the controller, the switch could attempt to overload it. One could fine-tune the control messages to maximize churn or CPU consumption in the controller.

However, our results show that a TCP congestion control successfully prevents a single switch from denying service to the controller; instead the extra packets from the switch are dropped.

What happens if in a network with n actual switches, the controller receives “hello” messages from $n^2$ switches, for example? This works around the throttling problem where a controller may simple rate-limit each switch.

\subsubsection{Dropping traffic} 
A compromised switch could simply drop packets sent to it, thus creating a service interruption. However, a controller would probably notice this quickly -- it would appear as though the switch had failed, and automated recovery mechanisms would be initiated.

\subsubsection{Cloning or diverting traffic}
A compromised switch could ensure that traffic would be routed through an adversary’s middlebox, or could clone traffic and send the cloned stream to the adversary. This would be more difficult for the controller to detect, unless it caused considerable slowdown. This attack could conceivably be used for purposes such as the NSA’s MUSCULAR program, which intercepts traffic as it flows between private data centers \cite{muscular}. 

\subsection{Unexecuted Attacks}
\subsubsection{DoS other switches}
By injecting traffic, a switch could deliberately attempt to overload neighboring switches' TCAMs or otherwise DoS them. The rogue switch could leverage its connection to the controller in order to prompt the controller to broadcast many erroneous flow modifications; thus, instead of attacking each network element individually, the controller gives the attacker an easy, cheap way to concurrently many switches at once. 

\subsubsection{Advertising false host attachments}
A compromised switch can send the controller packet-in events with false packets, which falsely claim to have a particular MAC address attached to them. This could lead the controller to believe that the compromised switch should receive traffic intended for the host. While this would soon lead to packet loss and a noticeable disruption, it would nevertheless allow a switch to eavesdrop when it would not normally be in a position to do so.

\subsubsection{Falsify measurement reports}
A switch may return false results in response to a read-state  measurement message, thus causing the controller to behave irrationally. For example, a switch could falsify or hide a DoS attack, elephant flows, etc.

\subsubsection{Ignoring rules}
A switch could simply ignore flow-table modification requests. For example, dropping packets will be noticed quickly; but a switch could allow packets to pass through that should have been dropped. Because of the difficulty of querying flow table state, the controller may not become aware of this. A compromised switch could thus operate in “stealth mode” and the inconsistent flow table might only be noticed once the switch allows a DoS attack to pass through, for example.

\subsubsection{Modifying VLAN tags}
A switch could modify VLAN tags in order to have malicious traffic be erroneously treated by other network elements. For example, if normal only packets with a certain VLAN tag are allowed to access a sensitive resource, the switch could ensure that the attacker's traffic appears to have valid permissions. % I'm really not so sure about this one, so maybe we shouldn't include it.

\subsubsection{Reporting flow mods to an adversary}
An adversary could hope to learn about network traffic by observing flow mods. Certain flow mods might indicate that the controller has, for example, detected an intrusion, or that a specific host has connected to the network in a certain location. Thus, just knowing what flow mods are being issued by the controller could be a source of interesting information for an adversary.

\subsubsection{Fingerprinting controller applications}
Similarly, the rogue switch could send packets directly to the controller and observe the controller's response in order to determine what kinds of applications are probably running on the controller. This information could then be used by an attacker to more narrowly target future attacks.

\subsection{Even more...}


\section{Related Work}
\label{related}
The related work might be long, so let's put it in a separate file, related.tex
This is related.tex - I've already added all the links from the email thread to the bibliography, we can remove the ones we won't plan to use.
TODO for Anna... look up the two security papers we read in class, add them to the bibliography and talk about them here.


\section{Conclusion}
The conclusion might be short enough to be in report.tex - we can add other sections before it as necessary, and the bibliography (in bibliography.tex) will follow.


% Imports the bibliography, don't remove
\begin{thebibliography}{99}

\bibitem{benton} Benton et. al. ``OpenFlow Vulnerability Assessment'' \emph{ACM SIGCOMM Hot Topics in Software Defined Networking (HotSDN)}, 2013. 

\bibitem{cisco} Cisco. ``Multiple Vulnerabilities in Cisco TelePresence Multipoint Switch'' 11 July 2012 \url{tools.cisco.com/security/center/content/CiscoSecurityAdvisory/cisco-sa-20120711-ctms}

\bibitem{ciscopatch} Constantin, Lucian. ``Cisco patches vulnerabilities in some security applications, switches, and routers.'' \emph{InfoWorld} 10 October 2013, \url{http://www.infoworld.com/d/security/cisco-patches-vulnerabilities-in-some-security-appliances-switches-and-routers-228551}

\bibitem{dpkt} dugsong et. al. ``dpkt'' \url{code.google.com/p/dpkt}

\bibitem{muscular} Gellman et. al. ``How we know the NSA had access to internal Google and Yahoo cloud data'' \emph{Washington Post, The Switch}, 4 November 2013, \url{http://www.washingtonpost.com/blogs/the-switch/wp/2013/11/04/how-we-know-the-nsa-had-access-to-internal-google-and-yahoo-cloud-data/}

\bibitem{hecker} Hecker, Artur. ``Carrier SDN: Security and Resilience Requirements'' 15 November 2013 \url{http://www.ikr.uni-stuttgart.de/Content/itg/fg524/Meetings/2013-11-15-Muenchen/12_ITG524_Muenchen_Hecker.pdf}

\bibitem{packetformat} ``OpenFlow Packet Format.'' \emph{OpenFlow}, \url{archive.openflow.org/wk/images/c/c5/Openflow_packet_format.pdf}

\bibitem{pycap} Rowe, Mark. ``Python Packet Capture and Injection Library'' \url{pycap.sourceforge.net}

\bibitem{attacking} Shin and Gu. ``Attacking Software-Defined Networks: A First Feasibility Study'' \emph{ACM SIGCOMM HotSDN}, 2013.

\bibitem{tcqack} Stretch, Jeremy. ``Understanding TCP Sequence and Acknowledgement Numbers'' 7 June 2010, \url{http://packetlife.net/blog/2010/jun/7/understanding-tcp-sequence-acknowledgment-numbers/}

\end{thebibliography}


\end{multicols}

\end{spacing}

\end{document}
