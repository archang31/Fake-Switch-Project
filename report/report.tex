\documentclass[11pt, letterpaper, twocolumn, twoside]{article}
%\usepackage{epstopdf}
%\usepackage[pdftex]{graphicx}
%\usepackage[left=.8in,top=.8in,right=.8in,bottom=.8in,nohead,nofoot,columnsep=20pt]{geometry}
\usepackage[left=1in,top=1in,right=1in,bottom=1in,columnsep=20pt]{geometry}
\usepackage{setspace}
\usepackage[small,compact]{titlesec}
\usepackage{subfigure}
%\usepackage{multicol}
%\usepackage{multirow}
\usepackage{textcomp}
\usepackage{mathtools}
\usepackage{amsmath}
\usepackage[hyphens]{url}
\usepackage{float}
%\usepackage{fancyhdr}
%\setlength{\headheight}{14.5pt}
%\setlength{\footskip}{11pt}
%\pagestyle{fancy}
%\lhead{}
%\chead{}
%\rhead{}
%\rfoot{}

\makeatletter
\setlength{\@fptop}{0pt}
\makeatother


\title{Flip that Switch! Implications of a Rogue Switch in SDN Security}
\author{Bonnie Eisenman, Michael Kranch and Anna Kornfeld Simpson}
\date{COS 597E Software Defined Networks \\ 14 January 2014}

\begin{document}

\maketitle

\begin{spacing}{1.0}

\begin{abstract}
This paper examines security concerns and potential vulnerabilities in Software Defined Networking (SDN). In particular, we look at the attack vector of remotely introducing a malicious switch into the networking path, something not possible without physical access to the switching site in a traditional network. We also look at the increased risk to the entire network from a single, compromised switch due to the centralized nature of SDN. We implement a rogue switch and execute a generalized denial of service attack on the controller without regard to the controller's applications or interaction mode (reactive or proactive). We then discuss several other attacks possible from a malicious switch and provide some recommendations for hardening SDN controllers and limiting these attacks.

%---- If we're putting this signposting in our intro, I'm not sure we also need it so explicitly here. Kept just in case. First, we provide an overview of SDN, the current status of security in SDN controllers, and discuss the threat models for our attack. We then introduce our rogue switch and present several attacks on various controllers. We also examine additional vulnerabilities to a SDN from a single compromised switch. Finally, we conclude with some recommendations for hardening SDN controllers and limiting the area of attack.
\end{abstract}

\section{Introduction}

Paragraph about SDN - the various meanings of SDN and that we are discussing the traditional controller switch networking.

Not sure of difference between intro / background anymore as Anna mentioned Friday.

Things to discuss
-- Controllers
----Various Controller. That we are doing testing on Ryu and Pox (possibly OpenDaylight but its more complicated)
----Switches - In this context, a switch means either an OpenFlow capable hardware switch or a completely software defined switch (OpenVSwitch)
----Need to Discuss Proactive verses Reactive Flow Installation - While the majority of the attacks presented in this paper would be more effective in a reactive Flow Installation configuration, they would still be 

Threat Model
- Need to make sure we specifically discuss current security in switch - using TLS certs, who supports, etc.
In section 2, we will.... In Section 3, etc. In Section 4, etc. Finally


\section{Creating a Software Defined Switch}
\label{fake}
In this section, we present our roque switch and 

\subsection {OpenFlow Protocol}
OpenFlow is the protocol used to communicate between the controller and its switches. An OpenFlow packet header is simply an 8 byte packet with the first byte used to communicate version, the second byte for type, third and fourth byte for message length followed by a four byte Transaction ID (see Figure 1). The type is of a subset of 19 possible types, most of which are of type request (e.g. 16 = Stats Request) with the subsequent reply (e.g. 17 = Stats Reply). There are couple key point to note when dealing with the OpenFlow protocol. First, the transaction ID is used to correlate incoming OpenFlow messages with their appropriate responses much in the same way TCP uses SYN and ACK flags. For example, a features reply will (generally) not be accepted by the controller unless it contains the corresponding transaction ID from the features request. This protocol is also a two-way communication scheme and not simply switch replies to controller request. Several message types, to include packet input events, are switch initiated communications to the controller.  

\begin{figure}
  \includegraphics[width=\linewidth]{openflowProtocol.png}
  \caption{OpenFlow Protocol Header Format \cite{protocol}}
  \label{fig:protocol}
\end{figure}

\subsection {Controller Connection Sequence}
While the specific initiation sequence varies by controller, there are several required command to initiate a switch to controller connection\footnote{We specifically tested our roque switch on the Pox, Ryu and OpenDaylight controllers. Based on our testing and the OpenFlow specifications, we believe all controllers require this small subset of initialization commands but differences might exist based on implementation}. The switch to controller connection simply starts by opening a TCP connection to the controller on the configured port (default 6634)\footnote{As previously discussed, there is no authentication for this connection supported by any tested controller outside of certificate authentication only utilized with OpenVSwitch}.  After opening the connection, the switch sends a Openflow Hello message (0) to the controller. The controller then responds with a Hello message and a Features Request (5) message to which the switch responds with a  Features Reply (6). The controller then sends a Set Config message (9) that does not trigger a reply. Ryu follows this message with a Barrier Request (18), and OpenDaylight follows it with a  Get Config Request (7) to ensure switch configs are set appropriately. OpenDaylight also sends a Flow Modification message (14) to delete any previously installed flows on the switch.

The above commands are all that is needed for the initial controller to switch connection. The switch then proceeds to send several link layer neighbor discovery protocol messages as Packet Input Notification (10) messages to the controller including  Neighbor Solicitation and Router Solicitation messages. The switch also sends s Multicast Listener Report (part of the Multicast Listened Discovery Protocol Version 2) to the controller. Finally, the switch begins periodically (approximately every 3 seconds) sending an OpenFlow Echo Request (2) to the controller as a form of a keep alive with the controller.

\subsection{Our Rogue Switch Utility}
Our Roque Switch Utility mimics the initial connection sequence to a controller in order to facilitate testing of switch and controller vulnerabilities. Our roque switch utility does not handle the actual routing of traffic and is instead simply used to disrupt controller and switch traffic flow\footnote{A more advanced utility including routing is discussed in our additional attack section as well as our future work}. The bulk of our utility is a OpenFlow message parser that handles and appropriately responds to controller messages in the correct message format. These messages are generally hardcoded mimicked responses from previously captured live switch communication with minimal dynamic pieces (the transaction id is dynamically assigned for instance)\footnote{Both Pox and Ryu do not actual verify that the transaction ID is correct on most messages. We were able to completely connection to both a Pox and Ryu controller simply my resending previously captured packets. Pox only verifies the transaction id on the Barrier Message of the initial config messages, and this reply can be ignored without the controller ending the TCP session}. 


\section{Disrupting an SDN with a Fake Switch}
\label{attacks}

\subsection{Executed Attacks} 

We conducted several attacks on various controller utilizing our rogue switch utility. 

 \subsubsection{DOS on the Controller}
We utilized the basic layer 2 mac address learning example from each controller with a modified the hard flow timeout of 1 as our setup\footnote{We used l2\_learning as our Pox controller and l2\_switch as our Ryu controller.}. We also utilize the basic mininet setup (1 switch with 2 hosts) to test the delay caused by our attack on each controller. We would have h1 ping h2 To test the delay cause on the controller, we would simply A more complicated controller (e.g. one that does more than simply add the mac address as a flow) would see its performance decrease more significantly than in our example test cases.

 The first attack we attempted was to see if a single switch could overload the controller by rapidly pushing information. We utilized the basic Echo Request (the switch keep-a-live message) as the OpenFlow message to send to the controller to establish a baseline for testing. Bigger packets, particularly ones that require some sort of computation by the controller application, would increase the performance effect on the controller. We started by sending a Echo Request as quickly as possible to the controller without caring about the response. This attack had little effect of the controller and simply resulted in periodic "TCP Previous Segment Lost" messages as a response from the controller. The issue with this method was we were never reading in the Echo Replies sent from the controller and the controller was simply ignoring our messages because its send buffer was full [I BELIEVE THIS IS CORRECT NOT 100]. We then modified this attack to send a single Echo Request, wait for an Echo Reply and then instantly send another. This attack caused the controller to continually send Echo Replies but at a controller specified rate - we had to wait for a reply before sending another request. As a result, this attack also had little effect on the controller

   To increase the effect of our attack the controller, we created multiple rogue switches and utilized the same attack on the controller. Our testing machine was actually the limiting factor in the number of rogue switches we could spawn, but we utilized 400 switches as the max for comparison purposes. These results (shown in Figure 2) show that we can severely effect a controller's response time simply by utilizing Echo Request. We further optimized our attack by splitting our TCP socket into a distinct listener and sender. We threaded our application so the two actions were not dependent on each other. With the optimized version, a single switch could fairly significantly lower the performance of the switch. We then tested our optimized attack against our test controller and noticed an even more significant result. When testing with 10 switches, the roque switches 
 too
By sending requests to the controller, the switch could attempt to overload it. One could fine-tune the control messages to maximize churn or CPU consumption in the controller.

However, our results show that a TCP congestion control successfully prevents a single switch from denying service to the controller; instead the extra packets from the switch are dropped.

What happens if in a network with n actual switches, the controller receives “hello” messages from $n^2$ switches, for example? This works around the throttling problem where a controller may simple rate-limit each switch.

\subsubsection{Dropping traffic} 
A compromised switch could simply drop packets sent to it, thus creating a service interruption. However, a controller would probably notice this quickly -- it would appear as though the switch had failed, and automated recovery mechanisms would be initiated.

\subsubsection{Cloning or diverting traffic}
A compromised switch could ensure that traffic would be routed through an adversary’s middlebox, or could clone traffic and send the cloned stream to the adversary. This would be more difficult for the controller to detect, unless it caused considerable slowdown. This attack could conceivably be used for purposes such as the NSA’s MUSCULAR program, which intercepts traffic as it flows between private data centers \cite{muscular}. 

\subsection{Unexecuted Attacks}
\subsubsection{DoS other switches}
By injecting traffic, a switch could deliberately attempt to overload neighboring switches' TCAMs or otherwise DoS them. The rogue switch could leverage its connection to the controller in order to prompt the controller to broadcast many erroneous flow modifications; thus, instead of attacking each network element individually, the controller gives the attacker an easy, cheap way to concurrently many switches at once. 

\subsubsection{Advertising false host attachments}
A compromised switch can send the controller packet-in events with false packets, which falsely claim to have a particular MAC address attached to them. This could lead the controller to believe that the compromised switch should receive traffic intended for the host. While this would soon lead to packet loss and a noticeable disruption, it would nevertheless allow a switch to eavesdrop when it would not normally be in a position to do so.

\subsubsection{Falsify measurement reports}
A switch may return false results in response to a read-state  measurement message, thus causing the controller to behave irrationally. For example, a switch could falsify or hide a DoS attack, elephant flows, etc.

\subsubsection{Ignoring rules}
A switch could simply ignore flow-table modification requests. For example, dropping packets will be noticed quickly; but a switch could allow packets to pass through that should have been dropped. Because of the difficulty of querying flow table state, the controller may not become aware of this. A compromised switch could thus operate in “stealth mode” and the inconsistent flow table might only be noticed once the switch allows a DoS attack to pass through, for example.

\subsubsection{Modifying VLAN tags}
A switch could modify VLAN tags in order to have malicious traffic be erroneously treated by other network elements. For example, if normal only packets with a certain VLAN tag are allowed to access a sensitive resource, the switch could ensure that the attacker's traffic appears to have valid permissions. % I'm really not so sure about this one, so maybe we shouldn't include it.

\subsubsection{Reporting flow mods to an adversary}
An adversary could hope to learn about network traffic by observing flow mods. Certain flow mods might indicate that the controller has, for example, detected an intrusion, or that a specific host has connected to the network in a certain location. Thus, just knowing what flow mods are being issued by the controller could be a source of interesting information for an adversary.

\subsubsection{Fingerprinting controller applications}
Similarly, the rogue switch could send packets directly to the controller and observe the controller's response in order to determine what kinds of applications are probably running on the controller. This information could then be used by an attacker to more narrowly target future attacks.

\subsection{Even more...}


\section{Countermeasures}
\label{countermeasures}
\subsection{Security as a Priority}

OpenFlow v1.0.1 currently lists SSL/TLS as an "optional" feature. Though previous security analyses have consistently recommended SSL/TLS as a solution to most vulnerabilities, neither controller developers nor hardware manufacturers have implemented SSL/TLS despite these glaring problems. This is in many ways a chicken-and-egg problem. Manufacturers can argue that there is no demand for extra security features; meanwhile, any potential OpenFlow users who need security won't bother with the platform.

Therefore, we recommend that future versions of the specification must include security as a top-level priority, with features such as SSL declared as mandatory, in order for us to have any hope of being able to secure OpenFlow systems. 

\subsection{Trust On Boot}

In the absence of SSL/TLS,  a controller can still attempt to protect itself from deceptive switches. The controllers we examined automatically accepted connections with our rogue switch without complaint. This default behavior is extremely dangerous, as any malicious attacker with the controller's IP address can easily connect to the controller and thereby execute any of the attacks from this paper. A controller could mitigate this risk by automatically accepting switch connections only for a short time window after boot; afterwards, new switches would have to be verified by some other mechanism--perhaps a whitelist established by the network administrator. However, this approach has the drawback of requiring extra work on the part of the network administrator.

\subsection{Dynamic Switch Inspection}

A controller could monitor switches' behavior and detect when a switch was behaving oddly. For example, one of our successful attacks involved rapidly migrating MAC addresses between fake switches. Detection algorithms could be trained on historical data, and identify such attacks as they happen. The suspicious switch could then be quarantined or removed from the network.

\subsection{Physical Security}

None of these attacks are possible without a way to initiate a TCP connection with the controller. The controller's IP address should be closely guarded, and 

\section{Related Work}
\label{related}
This is related.tex - I've already added all the links from the email thread to the bibliography, we can remove the ones we won't plan to use.
TODO for Anna... look up the two security papers we read in class, add them to the bibliography and talk about them here.


\section{Conclusion}
While software defined networking provides many new and exciting avenues for networking innovation, it also introduces new attack vectors. Our results show that a single compromised switch could significantly impair the performance of the controller and, consequently the entire network even if TLS encryption was used to protect the switch-to-controller communication. 

The grand tradition of SDN as a clean-slate networking design \cite{4d} \cite{ethane} \cite{sane} initially included a greater focus on security, particularly of enterprise networks, as one of the main motivations for the new paradigm.  Although this focus has waned in the rush to develop standards, controllers, programming languages, and keep up with industry adoption pressures, we hope that recent work in this area will re-invigorate work in secure SDN design.

% Imports the bibliography, don't remove
\begin{thebibliography}{99}

\bibitem{benton} Benton et. al. ``OpenFlow Vulnerability Assessment'' \emph{ACM SIGCOMM Hot Topics in Software Defined Networking (HotSDN)}, 2013. 

\bibitem{cisco} Cisco. ``Multiple Vulnerabilities in Cisco TelePresence Multipoint Switch'' 11 July 2012 \url{tools.cisco.com/security/center/content/CiscoSecurityAdvisory/cisco-sa-20120711-ctms}

\bibitem{ciscopatch} Constantin, Lucian. ``Cisco patches vulnerabilities in some security applications, switches, and routers.'' \emph{InfoWorld} 10 October 2013, \url{http://www.infoworld.com/d/security/cisco-patches-vulnerabilities-in-some-security-appliances-switches-and-routers-228551}

\bibitem{dpkt} dugsong et. al. ``dpkt'' \url{code.google.com/p/dpkt}

\bibitem{muscular} Gellman et. al. ``How we know the NSA had access to internal Google and Yahoo cloud data'' \emph{Washington Post, The Switch}, 4 November 2013, \url{http://www.washingtonpost.com/blogs/the-switch/wp/2013/11/04/how-we-know-the-nsa-had-access-to-internal-google-and-yahoo-cloud-data/}

\bibitem{hecker} Hecker, Artur. ``Carrier SDN: Security and Resilience Requirements'' 15 November 2013 \url{http://www.ikr.uni-stuttgart.de/Content/itg/fg524/Meetings/2013-11-15-Muenchen/12_ITG524_Muenchen_Hecker.pdf}

\bibitem{packetformat} ``OpenFlow Packet Format.'' \emph{OpenFlow}, \url{archive.openflow.org/wk/images/c/c5/Openflow_packet_format.pdf}

\bibitem{pycap} Rowe, Mark. ``Python Packet Capture and Injection Library'' \url{pycap.sourceforge.net}

\bibitem{attacking} Shin and Gu. ``Attacking Software-Defined Networks: A First Feasibility Study'' \emph{ACM SIGCOMM HotSDN}, 2013.

\bibitem{tcqack} Stretch, Jeremy. ``Understanding TCP Sequence and Acknowledgement Numbers'' 7 June 2010, \url{http://packetlife.net/blog/2010/jun/7/understanding-tcp-sequence-acknowledgment-numbers/}

\end{thebibliography}



\end{spacing}

\end{document}
