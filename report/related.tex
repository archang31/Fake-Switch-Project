Vulnerabilities in routers and switches are security challenges that predate SDN.  Hardware switches are typically proprietary, so customers are at the mercy of the company to identify problems and perform patches in a reasonable timeframe \cite{cisco}\cite{ciscopatch}.  In the meantime, actors of all types can use these vulnerabilities to gain access to otherwise protected network traffic, potentially without any detection: the NSA's ``SpyMall'' catalog contains some of these exploits \cite{spymall}.

As Kreutz et. al. noted in August 2013, ``security and dependability of SDN still is a field almost unexplored'' \cite{sdnsec} and this has not changed in six months.  However, their study provides important motivation for our work by identifying a compromised switch as a threat with an augmented importance in an SDN context.  Also important is the assessment by Benton et. al., particularly the discovery that secure communication via TLS is not even implemented in most switches or controllers \cite{benton}.  Another recent work in the area comes from Shin and Gu, who discuss Denial of Service attacks on the control and data planes by generating malicious flow requests \cite{attacking}.  Although their method does not involve a compromised switch, it appears that some of the proposed countermeasures for a malicious switch, particularly the use of TLS, could significantly reduce the feasibility of this attack.
